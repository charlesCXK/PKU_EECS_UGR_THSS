\chapter*{摘要}
% 这里写中文摘要
Copyright (c) 2019 Bochen Tan, Xiaokang Chen

Public domain.

本模板的宗旨是尽量绿色,不需要附加安装任何东西。

按照教务部下发的WORD说明文档格式,下简称“说明”

没有封面和评阅表,这两部分请直接在Cover\&ReviewTable.doc中写再输出pdf拼到一起

doc小改动:封面校徽和文字替换为了高清版本,“题目:”和中文题目对齐,中英文题目分在了表的两行

doc小改动:插入了两个白页,使得连续打印的时候封面和表格都在奇数页

正文部分改动:在每一页下方中央加了页码,因为说明中页眉不分奇偶页,所以页码就都在中央吧

不含自动的参考文献,说明中参考文献格式不典型,请手动输入或自行写程序

在Windows或Linux下渲染出字体更接近说明,Mac OS上字体不太一样

有警告$\backslash$ headheight is too small,fancyhdr的上距离有点小,似乎问题不大

\bigskip
\noindent{\bfseries\songti 关键词: }



\addcontentsline{toc}{chapter}{摘要} %手动加入目录
\fancypagestyle{plain} %因为latex默认每章第一页是plain所以需要重置一下plain和说明统一
{
	\fancyhf{} %清空

	\fancyhead[RE,RO]{摘要}
	%偶数页右页眉,奇数页右页眉均为“摘要”,及章名\leftmark

	\fancyhead[LE,LO]{北京大学本科生毕业论文}
	%偶数页左页眉,奇数页左页眉均为“北京大学本科生毕业论文”

	\fancyfoot[CO,CE]{~\thepage~}
	%偶数页和奇数页中页脚为页码,从对称考虑,因为每页在说明中都是一样的,不分奇偶

	\renewcommand{\headrulewidth}{0.7pt} %页眉线宽度,可调,不太清楚说明中是多少,待改

	\renewcommand{\footrulewidth}{0pt} %页脚线宽度为0,既没有
}

%默认的风格是fancy,设置于下,用于每章非第一页
\fancyhf{}
\fancyhead[RE,RO]{摘要}
\fancyhead[LE,LO]{北京大学本科生毕业论文}
\fancyfoot[CO,CE]{~\thepage~}
\renewcommand{\headrulewidth}{0.7pt}
\renewcommand{\footrulewidth}{0pt}
\clearpage